%===================================== CHAP 1 =================================
\chapter{Introduction}

\section{Background and Motivation} \label{sec:background}
The Construction Industry (CI) has been a significant part of engineering throughout history. Over the past century, the requirements of constructions have become more and more complex \cite{wood2009factors}. The buildings are getting higher, the tunnels are getting longer, and the roads are getting wider. Sure, the size of things is not equal to the complexity of the construction; however, when considering automated systems, multipurpose functionality, and multiple communication platforms, alongside multiple new roles – the complexity is increasing \cite{arayici2010building}. The increased complexity leads to a significant decline in labor productivity (LP), seen over the past two centuries, mentioned in the article written by SSB \cite{productivity}. As well, managing these projects is much more intricate then it used to, because of the increased numbers of actors participating in the project. 
 
One can argue that the negative progress in LP in the CI has to do with the increasing complexity, and therefore not a number to consider. Even so, better productivity and efficiency are always something management dicier, simply because of improved marginal cost. Therefore, this study is interesting for managers from other industries than only construction.
 
One has often turned to software when wanting to improve productivity. So is the case in the CI as well, moreover, changing the working process, and working methodology. The implementation of software in a large, complex organization is discussed by many and is shown hard to do adequately. In particular, the description of the top-down contra bottom-up strategy in implementation \cite{Robey&Sahay}, promotes the importance of making slow change supported by the users. Furthermore, the intention of increasing productivity, by deploying new software is argued by Hammer, to be less sufficient \cite{hammer1990reengineering}. Hammer promotes changing the process of work, rather than improving bits by pieces using specific software. Moreover, the introduction of software supporting collaboration is challenging. Relevant in this context is that software breaks with the social taboos, and adaptation is, as mention, difficult \cite{Grudin}. 
 
Frank Garry, in 1997, first introduced 3-D modeling in CI, when constructing the Peter B. Lewis Building (PLB). 3-D modeling was introduced both in managing the complicated installation, but also led to increased cooperation between different parties within the project. The paper, describing this project \cite{frank_gehry}, is reporting a change in how actors in the construction react to using computer-aided constructions, in 3-D. Today 3-D modeling is used in almost all construction projects and is known as Building Information Modeling (BIM). Even though the PLB-project showed promising results in means of cooperation and interaction, the introduction of 3-D modeling was not a single solution to the problem. 
 
Using BIM and ICT-solutions has previously been the driving force of facing the aforementioned issues \cite{arayici2010building}, but seeing how this initiative changes the construction process is, hence, important. Furthermore, one has introduced Lean in the CI. A book \cite{lean_i_praksis} describing the making of the Bergen Academy of Art and Design-building, where Lean was one of the essential strategies. The case object of the case study in this research is using experience from this book when managing the constructions. 
 
The motivation for this research is, therefore, to examine a construction project utilizing Lean in project management. Furthermore, looking at how a project makes use of digital tools, aiding Lean has not been examined before. Taking experience from the ICT-industry, and the use of computer-aided agile development management is also desirable, as well as looking at the problem from a different perspective.
 
The case object chosen for this project is the construction of the new life science building. The reason for choosing this project is, first, the construction is highly complex, becoming the most extensive educational building in Norway, with both advanced technical and environmental requirements. Second, the announced strategies for the project includes both a lean strategy and a digitalization strategy. 

The conduction of this research was done throughout two phases. The first phase, conducted in the fall of 2019, resulted in a project thesis, while this thesis is the result of both phases, including data from the whole period.

\section{Research and Question} \label{sec:research}
Based on the background and motivation the research of this project tries to identify the phenomenon causing the poor PL in CI. The main research question is, therefore:

\begin{quote}
    \textit{What is the fundamental reason for the striking issue of labor productivity in the construction industry?}
\end{quote}
This is then broken down to two sub-questions, which this project thesis tries to answer, using a case study of the Life Science Building project.

{\noindent \bf RQ1:} Why does the issue of labor productivity in the construction industry appear? \\
{\noindent \bf RQ2:} How does this issue appear in the LSB-project, which utilize both agile and digital tools?

\section{Thesis structure} \label{sec:thesis}

{\noindent \bf Chapter 2: Literature Review} provides an overview of key findings, concepts and development relevant for the research question. Furtheremore, support the discussion as well as the case. 

{\noindent \bf Chapter 3: Empirical Review} gives an introduction of the case, and impede how they make use of digitalization in their Lean Strategy, as well as how this support cooperation and knowledge sharing in the organization.

{\noindent \bf Chapter 3: Method description} describe the mothodology used in the project. The methodology descriptin describe and discuss the approach, data collection as well as method of anaylysis of the genereated data. Also, an evealuation of the method i provided.

{\noindent \bf Chapter 4: Case review} gives an introduction of the case as context for the project. Furhteremore, descibe and discuss the result of the anaylysis of the case data. 

{\noindent \bf Chapter 4: Discussion} takes the data form the case study, and discuss the results with prior research identified in the literature review. The chapter is outlined by three themes discovered in the analysis of the case data. Furthermore, the evaluation of the project limitation can be found in this chapter.

{\noindent \bf Chapter 5: Conclusion and Further work} answer the research questions raised in the \nameref{sec:research} section. Furtheremore, proposing furhter work for the master thesis.


\cleardoublepage