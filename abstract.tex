\clearpage
\pagenumbering{roman} 				
\setcounter{page}{1}

\pagestyle{fancy}
\fancyhf{}
\renewcommand{\chaptermark}[1]{\markboth{\chaptername\ \thechapter.\ #1}{}}
\renewcommand{\sectionmark}[1]{\markright{\thesection\ #1}}
\renewcommand{\headrulewidth}{0.1ex}
\renewcommand{\footrulewidth}{0.1ex}
\fancyfoot[LE,RO]{\thepage}
\fancypagestyle{plain}{\fancyhf{}\fancyfoot[LE,RO]{\thepage}\renewcommand{\headrulewidth}{0ex}}

\section*{\Large Abstract}
\addcontentsline{toc}{chapter}{Abstract}

\noindent A primary concern of the construction industry in Norway is the significant decline in labor productivity compared to other on-land industries. Statistics show a relative difference in labor productivity, compared to the ICT industry from the year 2000 up to 2016, of 106.4\%. The research identifies an industry utilizing old methods, with the use of new technology. The change in technology is causing new roles and more complex constructions—this change made for utilizing agile management and software supporting these methods. This case study looks at the new life science building project, of the University of Oslo, to explore the primary conditions for Norwegian construction projects, utilizing Lean and BIM, to achieve the potential of both the applied methodology and digital tools.

Using semi-structured interviews and observations, with participants from all disciplines from the project organization, a thematic analysis of the interviews identified two key problems: (1) \nameref{sec:unmitigated} and (2) \nameref{sec:lack_of_knowledge}. The research identified a challenge with the use of different software in the project, making the actors using other tools - hence, the problem of overlapping tools. Furthermore, the project does not utilize the potential of applied BIM technology.

Based on these observations, the project proposes two new initiatives for better the utilization of Lean and applied software in the construction industry. First, new research is needed to find a method of writing task descriptions used in the Lean Design process. Moreover, investigate the need for such a framework in Lean Construction. Second, more research is needed in using cloud computing in BIM modeling.

\clearpage