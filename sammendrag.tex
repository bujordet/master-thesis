\clearpage
\pagenumbering{roman} 				
\setcounter{page}{1}

\pagestyle{fancy}
\fancyhf{}
\renewcommand{\chaptermark}[1]{\markboth{\chaptername\ \thechapter.\ #1}{}}
\renewcommand{\sectionmark}[1]{\markright{\thesection\ #1}}
\renewcommand{\headrulewidth}{0.1ex}
\renewcommand{\footrulewidth}{0.1ex}
\fancyfoot[LE,RO]{\thepage}
\fancypagestyle{plain}{\fancyhf{}\fancyfoot[LE,RO]{\thepage}\renewcommand{\headrulewidth}{0ex}}

\section*{\Large Sammendrag}
\addcontentsline{toc}{chapter}{Sammendrag}

\noindent En bekymring for bygg- og anleggsvirksomheter i Norge er den betydelige nedgangen i arbeidsproduktivitet sammenlignet med andre Fastland industrier. Statistikk viser en relativ forskjell i arbeidsproduktivitet, sammenlignet med IKT-industrien fra år 2000 og fram til 2016, på 106,4 \%. Forskningen identifiserer en industri som bruker gamle metoder, men som bruker ny teknologi. Endringen i teknologi forårsaker nye roller og mer komplekse konstruksjoner - denne endringen har ført til en innføring smidig metoder og ulike programvarer som støtter disse metodene. Denne studien skal se på det nye Livsvitenskaps byggeprosjektet, ved Universitetet i Oslo, for å utforske betingelsene for norske byggeprosjekter, som bruker både Lean og BIM, for å oppnå potensialet i både anvendt metodikk og digitale verktøy.

Ved bruk av semistrukturerte intervjuer og observasjoner, med deltakere fra alle fagfelt fra prosjektorganisasjonen, identifiserte en tematisk analyse av intervjuene to viktige problemer: (1) overlappende programvare funksjonalitet og bruk, og (2) Mangel på grunnleggende metodisk kunnskap. Forskningen identifiserte en utfordring med bruk av forskjellig programvare i prosjektet, noe som gjorde at aktørene brukte andre verktøy - derav problemet med overlappende verktøy. Videre utnytter ikke prosjektet potensialet til anvendt BIM-teknologi.

Basert på disse observasjonene foreslår prosjektet to nye initiativer for bedre utnyttelse av Lean og anvendt programvare i byggebransjen. For det første trengs ny forskning for å finne en metode for å skrive oppgavebeskrivelser som brukes i Lean Design-prosessen. Undersøk dessuten behovet for et slikt rammeverk i Lean Construction. For det andre er det behov for mer forskning når det gjelder bruk av sky-tjenester i BIM-modellering.
\clearpage