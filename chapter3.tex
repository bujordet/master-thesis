%===================================== CHAP 3 =================================

\chapter{Method} \label{chp:method}
This chapter will give an insight into why the research is needed and what method used conducting the research. Also, the chapter gives insight into the collection of data and the following analysis, as well as and who are the participants.

\section{Methodological Approach} \label{sec:purpose}
This study aims to understand the fundamental reason for the striking challenge in labor productivity in the CI-industry. The research is, therefore, adopting a case study-strategy of a single-case object, in the CI in Norway. Utilizing a single-case study approach, preferably than multiple, will give a more in-depth look at the problem, rather than a thin description provided by the multiple-case study \cite{yin1993case}. This project, therefore, aims to examine a case using lean methodology, utilizing digital tools to support both the method as well as cooperation and interaction between different actors. The project selected is the construction of the new Life Science Building, managed by Statsbygg \cite{statsbygg2019uio}. Obtaining the LSB-project was rather by chance, and followed no formal theoretical sampling procedures proposed by the literature \cite{yin1993case}. The problem of using a case study is that it is hard to produce a generalized answer to a question. The aim of the research is not to obtain generalizable findings but to explore the phenomenon. Furthermore, identify different measures that can help this specific project. This thesis is based on a preliminary project committed in the fall of 2019. The intention is not to measure productivity, but rather understand the phenomena and propose coherent actions.

This study is related to the interpretivism paradigm — the use of empirical observation of the participants and a desire to identify how they act on the new software and methods used. Using interviews can lead to being subjective as all collection of data is done in interaction with the participants. This yields a qualitative collection of data. The purpose of this project thesis is to identify issues causing a lack of productivity and identify actions fixing these issues. 

Due to the COVID-19 virus, this thesis could not implement the identified actions, and therefore only propose the actions.

{\noindent \bf RQ1:} Why does the lack of productivity in labor productivity in the construction industry appear? \\
{\bf RQ2:} How does this problem appear in the LSB-project, which utilizes both agile and digital tools?

\section{Data Collection}
\begin{figure}
    \begin{center}
        \includegraphics[width=0.75\textwidth]{fig/research_process_master.png}
        \caption{The research process used, marked with methods applied in the research.}
        \label{fig:research_process_master}
    \end{center}
\end{figure}

This project is following a preliminary study, resulting in a project thesis, in the fall of 2019. Starting with the literature review, providing a knowledgeable background as the basis for the project. Moreover, a minor empirical study of the case object, including both interviews and observations. This thesis consists of a more in-depth empirical study, with several interviews as data generators. 

\subsubsection{Interviews}
The interviews for this research was done during two phases, as seen in the \ref{tab:participants}. 

The first phase of the research, resulting in a project thesis, was conducted in the fall of 2019. Due to a hectic period of the project, the researcher made use of video chat conducting the interviews. Using Skype and other video chat services can be cost-effective due to the ease of planning, compared to face-to-face interviews. On the other hand, because of the small window the web camera provides, it could be challenging to read the participant's body language \cite{cater2011skype}.  Also, the utilization of digital interviews is reliant on highspeed internet, and that the subject is familiar with the digital tool for the interview to work well enough.

In the second phase, the interview took place at the project office, near the building site, in Oslo. The project gave up a small meeting room for a week for the researcher to use. Opposite of digital conduction, face-to-face interviews lead to more waste of time caused by the difficulty of planning and waiting. On the other hand, the conversation will not suffer from a lack of body language or depend on technical tools to work, but the recording device. Also, the researcher feels the conversation had a better flow face-to-face. 

Every interview can be a source of error because of the communication element and how the actors interpret. What the interviewee sais is not only in words, but mimics, cadences, body language as well. Hence the difference in the face-to-face and video chat. Using follow-up questions helps in the case of insecurity. Also, the researcher could correspond with the participants over email and telephone after the interviews. Thus, increasing the validity of the data. 

Before an interview, an email was sent to the participant, preparing for the conversation. The email consisted of a description of the research, the researcher. Moreover, the consent and interview guide described later in this section. The preparation gave the researcher more in-depth and thought-through answers. Not every participant had spent time preparing for the interview. 

The conduction of the interviews followed a set of standardized questions. The researcher, beforehand, developed these questions. The guide consists of a small collection of themes relevant to the research question, listed in Appendix \ref{apx:interview_guide}. The order in which the questions where asked was not of importance, and often changed based on the interviewee—moreover, the participant was indeed the dominant part of the interviews. The interviews are, therefore, consisting of several follow-up questions. The purpose of an unstructured interview was for the interviewee to speak freely, and therefore more receive more genuine answers. The research process used is shown in figure \ref{fig:research_process_master}.

In line with Norsk Senter for Forskningsdata AS (NSD), the researcher, signed an informed consent with the participants. The contract gives the researcher allowance to do the interview; keep personal data throughout the project, such as name, title, and company; and do recording of the interview. The purpose of the recording was for the researcher not to take heavy minutes during the interview, instead focus on the conversation. Also, a recording will give an exact version of the interview. 

The transcription became more or less the exact recording of the interview. Sometimes, not writing the follow-up questions, if the sentence made sense. The resulting text was hard to read, due to its spoken tone of voice. The purpose was not to make a perfectly readable text, but to have documentation of the thoughts and experience of the subjects, for further analysis. Moreover, after the interview, the transcript was sent to the participants for them to approve. Every interview lasted between 15 minutes and up to half an hour.

\section{Participants}
The project researcher, Morten Bujordet, is involved in the project, creating the plans, and conducting the research.
	 
Supervising the project is Eric Monteiro. Monteiro is contributing with experience in research in the implementation and use of new digital tools in large scale, as welle as complex organizations. 

Furthermore, Statsbygg, as the manager, has an interest in the project: giving access to the participants in the study. With Darre Brecke Brenden as the point of contact.
	 
In the research, the actors in all layers and disciplines of the project organization will be an aim for the data collection. All personal information gathered will, safely, be stored in a GDPR-compliant Cloud Service, served by NTNU. In the final report, no personal information will be published, and all participants will be anonymized. The participants chosen for the interviews are key personnel leading, modeling, and working in the design of the project. Statsbygg gave a list of 30 candidates, where 16 had the opportunity to participate in the interviews. The interviewees are listed in table \ref{tab:participants}. 
 %TODO: Smelt sammen disse to tekstene
The thesis conducted a total of 18 interviews. Selecting the first two interview objects was based on a list of 6 interviewees, proposed by the first contact person. Only two subjects had the opportunity to participate, due to a hectic period of the project.  The second phase of interviews started with a list of 32 potential interview objects. Every proposed interviewee was contacted; only 16 had the opportunity. The initial list of persons was handpicked, by the assisting project director, to get an insight into all different levels of seniority and different disciplines of the project. The resulting set of interviews represent a wide range of project seniority. From a month of experience and up to the very start of the project, back in 2014. Also, the set represents every discipline in the project. 

\begin{table}
    \resizebox{\textwidth}{!}{%
    \begin{tabular}{@{}lllll@{}}
    \toprule
    \textbf{Interviewee} & \textbf{Function} & \textbf{Gender} & \textbf{\begin{tabular}[c]{@{}l@{}}Phase 1\\ (November 2019)\end{tabular}} & \textbf{\begin{tabular}[c]{@{}l@{}}Phase 2\\ (February 2020)\end{tabular}} \\ \midrule
        1 & \begin{tabular}[c]{@{}l@{}}Assistant Project Director \& \\ Project Manager \end{tabular} & Male & x & \\
        2 & Assistant Project Manager  & Male & x & \\ 
        3 & Project Manager & Male & & x \\
        4 & Engineering Manager & Male & & x \\
        5 & Engineering Manager & Male & & x \\
        6 & Progress Planner & Female & & x \\
        7 & ITB Manager & Male & & x \\
        8 & Associate & Female & & x \\
        9 & Discipline Leader & Male & & x \\
        10 & Discipline Leader & Female & & x \\
        11 & BIM Manager & Male & & x \\
        12 & Associate & Male & & x \\
        13 & Associate & Male & & x \\
        14 & Discipline Leader & Male & & x \\
        15 & Assiciate & Male & & x \\
        16 & Ass. Project Group Leader & Female & & x \\
        17 & Engineering Manager & Male & & x \\
        18 & BIM Coordinator & Male & & x \\
        Total interviews & \multicolumn{2}{r}{} & 2 & 16 = 18\\ \bottomrule
    \end{tabular}%
    }
    \caption{Overview of interviews and phases of data collection.}
    \label{tab:participants}
\end{table}

\section{Data Analysis}
The analysis started by transcribing the interviews and sent to the participants for approval. Furthermore, utilizing a thematic analysis approach. The thematic analysis followed a set of steps

\begin{enumerate}
    \item A perusal of all the material: It was essential to get to know the material, and to read through all the interviews after the transcription was a vital part of this step. The interviews was ; 
    \item Generation of codes: After getting to know the material it the focus was to identify parts of the text based on the project question, marked with an identifying code; 
    \item Collection of themes: Based on the codes, a set of themes evolved. The codes find most connected were put into groups, based on the connection between them. Which themes adopted in the thesis was based on the research question; 
    \item  Reviewing the themes: Based on the research question, a set of themes was selected. Basing the selection on different criteria, among others, how interesting the theme is, the iteration of the code, and the researcher's considerations; 
    \item Defining and naming the themes: The final collection of themes is selected, and will be the basis for the discussion in the rest of the thesis. The names of the themes are also selected. In this process, three themes appeared, namely: (1) Not unmitigated,(2) Lack of knowledge, and (3) digital potential; 
    \item Producing the report: Describing the findings in the thematic analysis, using the chosen themes as a guideline. Moreover, discussing the themes with relevant literature, previous experience, and context.
\end{enumerate}

The data produced in this process is in the Appendix. The initial codes identified Appendix \ref{apx:codes} and the collection of themes in Appendix \ref{apx:themes}.
Reviewing of themes did not produce much data, due to its subjective manifestation. 

\section{Evaluation of the Method}
The single-case study, as well as the use of unstructured interviews, produce results that cannot be generalized beyond the sample group. Still, they provide a more in-depth understanding of participants’ perceptions, motivations, and emotions. 

One can always argue that utilizing interviews for data collection can tend to be subjective. However, the use of a qualitative approach is best when wanting to describe, contextualize, and gain an in-depth insight into specific concepts or phenomena, which was the case in this empirical study. Furthermore, the project researcher has a part-time job developing the Cogito tool, which can argue for the researcher for being subjective. Though, in this case, 14 of 16 participants mentions Cogito, without the researcher asking them. Also, the participants did not know the relation the researcher had to Cogito; thus, the interviewees spoke freely. Moreover, based on the interviews, the Cogito tool was the one subject getting the most tension; therefore, discussing Cogito and themes related is arguably based on a valid reason. 

The objective of this thesis was to test some of the concluding proposals and observe the change it might bring. Though, due to the COVID-19 virus, the implementation phase was not feasible. Thus, this thesis only consists of a set of proposed actions. Testing of the actions is, therefore, up to a later project, or for the project to do. The project owner has received a summary of this thesis, including the proposed actions. 

Further exploration of the project question is needed to conclude on the matter, moreover, testing of the suggested actions.
\cleardoublepage