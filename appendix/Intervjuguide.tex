\chapter{Interview Guide} \label{apx:interview_guide}
\noindent\textbf{Intervjuguide} \\
Navn og rolle: \\
Dato: \\

\noindent\textbf{Avklaringer og informasjon:} 
\begin{itemize}
    \item Avklare anonymitet/åpenhet
    \item Informere intervjuobjektet at det vil bli tilsendt transkribert versjon i etterkant av intervjuet.
    \item Informere om bruk av lydopptak
    \item Kandidat for mulighet til å presentere seg selv og sin erfaring.
\end{itemize}

\noindent\textbf{Mitt prosjekt} \\
Prosjektets påstand er at byggebransjen i dag opplever mange av de samme symptomene som programvareutvikling opplevde for 30 år tilbake og stadig opplever. Utfordringene opplevd er: Stadig endring av requirements under produksjon, slite med å nå tidsplan og budsjett, økende kompleksitet – hvert prosjekt er noe helt nytt, og sist med ikke minst problemer med ferdigstillelse. Programvareutvikling har derfor tatt i bruk agil prosjektstyring og digitale verktøy for å løse flere av disse problemene.

Sett i lys av dette ønsker prosjektet (jeg) å se hvordan et byggeprosjekt med høy kompleksitet utnytter smidige metodikker (inkluder Lean), støttet av digitale verktøy i sin prosjekthverdag. \\

\pagebreak

\noindent\textbf{Temaer som ønskes belyst}
\begin{enumerate}
    \item Generelt om delegering, samarbeid og kunnskapsutveksling
    \begin{enumerate}
        \item Hvordan fungerer delegering, samarbeid og kunnskapsutveksling?
    \end{enumerate}
    \item Bruk av digitale prosjekteringsverktøy i prosjektet
    \begin{enumerate}
        \item Hvordan og hvilke digitale digitale prosjekteringsverktøy benyttes i\\ prosjektet, i dette henseende?
        \item Finnes det noen begrensninger ved denne bruken?
    \end{enumerate}
    \item Andre faktorer
    \begin{enumerate}
        \item Er det andre faktorer som påvirker bruken av verktøyene i prosjekthverdagen?
    \end{enumerate}
\end{enumerate}