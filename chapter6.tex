%===================================== CHAP 5 =================================

\chapter{Conclusion} \label{cha:conclusion}
The previous chapter analyzed and discussed the results of the research conducted in this project, as well as discuss prior research and experience identified in the literature study. This chapter aims to conceptualize the knowledge and results in a conclusion, answering the questions asked in section \ref{sec:rq1} and \ref{sec:rq2}. Furthermore, giving a direction for further explorations in section \ref{sec:further_research}.

\noindent The main research question of this projects was: 
\begin{quote}
    \textit{What is the fundamental reason for the striking issue of labor productivity in the construction industry?} 
\end{quote}

\noindent Which was broken into two sub-questions:

{\noindent \bf RQ1:} Why does the issue of labor productivity in the construction industry appear? \\
{\bf RQ2:} How does this issue appear in the LSB-project, which utilize both agile and digital tools?

Section \ref{sec:rq1} and \ref{sec:rq2} outline the answer of RQ1 and RQ2 accordingly.

\section{A Change in Paradigm} \label{sec:rq1}
The CI has become more and more digitalized. BIM has had a positive influence, both in cooperation and interaction \cite{frank_gehry, gu2010understanding}, but also visibility for the project workers \cite{arayici2010building}, as seen in \ref{sec:digital_potential}. Following this digitalization, a large set of new positions occurs; thus, the interaction and coordination get more complicated, which gives a more complex construction \cite{arayici2010building, eadie2013bim, arayici2011technology}.

There is no single answer to this sub-question, but the literature study identified an industry utilizing processes far behind its counterparts with an ever-increasing complexity hard to grasp with traditional processes. Proper communication and cooperation are vital for every project to be more productive. Notable is the legal contracts need to mirror new work methodologies and increasing interactions in the digital sphere. When following this change, this different system of working is hard to grasp for the older players of the game; thus, leading to poor utilization. 

Moreover, making every new project better, and more productive, will eventually have a positive impact on the overall LP of the IC. Also, implement one great project will not change the overall LP in the CI, but can indeed be a good influence on other projects.

\section{Problems Following a Change} \label{sec:rq2}
Based on a case study researching a project utilizing lean and digital tools. The project has utilized a qualitative analysis of the case study, the research has identified two themes explained in section \ref{sec:unmitigated} and \ref{sec:lack_of_knowledge}. Moreover, the discussing issues with the knowledge from the literature review presented in the \nameref{cha:discussion} chapter. These are the problems causing the challenge in LP, discovered in this project.
    
{\bf Overlapping software functionality and software usage:} The ever-increasing complexity has led to a significant number of new software supporting the construction process. Thus, in a current construction project every worker has to deal with numerous different tools, aiding the daily work. The problem is the overlapping functionality making for different use and challenge in cooperation between different disciplines.
    
{\bf Lack of fundamental methodological knowledge:} The project has implemented a set of new methods and processes. The new implementations require a basic understanding of Lean Construction; thus, training is mandatory for the team to utilize the methodology and supplied tools fully. Besides, the new methodology does not cover how to do all aspects of the process, e.g., writing task descriptions.

The LSB-project has done a great job utilizing both new contracts, a new methodology, and new software, which should help in managing a complex construction project. All literature indicates that the LSB-project becomes an influential project for the rest of the industry admire.

\section{Further Work} \label{sec:further_research}
Based on the results of the case study and review of the literature, this research has two main recommendations in further research.

First, further research in evolving a way of writing better task descriptions in a Lean Design process is prohibited. Moreover, investigate the need and difference in the construction phase, in Lean Construction. It will be beneficial using experience from both User stories from Scrum and how to write sound requirements from RE.
	 
Second, the results indicate a considerable benefit in utilizing cloud technology in BIM. The personalized tools, however, addresses a different approach, namely cutting the exportation barrier, connecting the already familiar tools directly to a shared cloud database. Thus, utilizing both the interaction benefit of cloud BIM, with the powerful tools used by the modelers. This enhancement implicates further development of the technology as well as research into the actual effect.

\cleardoublepage