%===================================== CHAP 5 =================================

\chapter{Conclusion} \label{cha:conclusion}
The previous chapter analyzed and discussed the results of the research conducted in this project, as well as discuss prior research and experience identified in the literature study. This chapter aims to conceptualize the knowledge and results in a conclusion, answering the questions asked in section \ref{sec:rq1} and \ref{sec:rq2}. Furthermore, giving a direction for further explorations in section \ref{sec:further_research}.

\noindent The main research question of this projects was: 
\begin{quote}
    \textit{What is the fundamental reason for the striking issue of labor productivity in the construction industry?} 
\end{quote}

\noindent Which was broken into two sub-questions:

{\noindent \bf RQ1:} Why does the issue of labor productivity in the construction industry appear? \\
{\bf RQ2:} How does the difference appear in the LSB-project, which utilize both agile and digital tools?

Section \ref{sec:rq1} and \ref{sec:rq2} outline the answer of RQ1 and RQ2 accordingly.

\section{RQ1} \label{sec:rq1}
There is no single answer to the question, but the literature study identified an industry utilizing processes far behind its counterparts with an ever-increasing complexity hard to grasp with traditional processes. Proper communication and cooperation are vital for every project to be more productive. Also, the contracts need to mirror today's work methodologies and increasing interactions in the digital sphere.
	 
Based on a case study researching a project utilizing lean and digital tools. The project has utilized a qualitative analysis of the case study, the research has identified two themes explained in section \ref{sec:unmitigated} and \ref{sec:lack_of_knowledge}. Moreover, the discussing issues with the knowledge from the literature review presented in the \nameref{cha:discussion} chapter. These are the problems causing the challenge in LP, discovered in this project.
    
{\bf Overlapping software and software usage:} The ever-increasing complexity has led to a significant number of new software supporting the construction process. Thus, a current construction project every worker has to deal with numerous different tools, aiding the daily work. The problem is the overlapping functionality making for different use and 
    
{\bf Lack of Fundamental Methodological Knowledge:} The project has implemented a set of new methods and processes. The new implementations require a basic understanding of Lean Construction; thus, training is mandatory for the team to utilize the methodology and supplied tools fully. 

\section{RQ2} \label{sec:rq2}

\section{Further Work} \label{sec:further_research}


\cleardoublepage