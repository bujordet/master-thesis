%===================================== CHAP 5 =================================

\chapter{Conclusion} \label{cha:conclusion}
The previous chapter analyzed and discussed the results of the research conducted in this thesis, as well as discuss prior research and experience identified in the literature study. This chapter aims to conceptualize the knowledge and results in a conclusion, answering the main research question in section \ref{sec:rq1}. Furthermore, giving a direction for further explorations in section \ref{sec:further_research}.

\noindent The main research question of this research was: 
\begin{quote}
    \textit{What are the primary conditions for Norwegian construction projects, utilizing Lean and BIM, to achieve the potential of both the applied methodology and digital tools?} 
\end{quote}

\noindent Which was broken into four sub-questions:

{\noindent \bf RQ1:} How does the project facilitate excellent communication and interaction as a basis for achieving the potential? \\
{\bf RQ2:} How is the project suited to meet the requirements in realizing the requirements of a construction process and Lean methodology? \\
{\bf RQ3:} How does the project realize its digital potential through the use of BIM and different groupware? \\
{\bf RQ4:} What are the challenges and conditions needed to be addressed in the project and future work? \\

Section \ref{sec:cooperation}, \ref{sec:methodology}, \ref{sec:digital_potential}, and \ref{sec:recommendations}, in chapter \ref{cha:discussion} outlines the answer of RQ1 through RQ4 accordingly.

\section{Challenges Following a Change} \label{sec:rq1}
Based on a case study of a Norwegian construction project, utilizing Lean and BIM. The research, investigating the Life Science Building project, emphasizes the importance of having a project structure fitting the applied methods and digital processes. This research has utilized a qualitative analysis of the case study; this identified two themes that caused challenges and impeded the potential:

{\bf Overlapping software functionality and software usage:} The ever-increasing complexity has led to a significant number of new software supporting the construction process. Thus, in a current construction project, every worker has to deal with numerous tools, aiding the daily work. The problem is the overlapping functionality making for different use and challenge in cooperation between different disciplines. Moreover, the utilization of the tools applied, such as in BIM, makes for lack of potential effect.  
    
{\bf Lack of fundamental methodological knowledge:} The project has implemented a set of new methods and processes. The new implementations require a basic understanding of Lean Construction; thus, training is mandatory for the team to fully utilize the methodology and supplied tools. This different system of working is hard to grasp when following this change for the older players of the game; this can lead to reduced utilization. Besides, the new methodology does not cover how to do all aspects of the process, e.g., writing task descriptions.

Disregarded the challenges above, the LSB-project has done a great job utilizing both new contracts, a new methodology, and new software to manage a complex construction project. While a single case study limits the generalizability of the results, this approach provides new insight into the generalized problem of setting up a project, meeting new requirements of the applied methodologies, and digital processes and tools. Moreover, due to Covid-19, the change in the execution of the research method makes for research without testing. Thus, the potential new implementations are for future work to investigate. 

The introduction introduced an industry challenged by a lack of LP. This project has shown a change in using both Lean and digital processes and tools. The intention is probably not directly for the project to defy the LP-challenge. However, making every new project better, and more productive will have a positive impact on the overall LP of the IC. Moreover, implement one great project will not change the overall LP in the CI, but can indeed be a good influence on other projects; hence, the project saying - \textit{"an even better project."}

\section{Further Work} \label{sec:further_research}
Based on the results of the case study and review of the literature, this research has two main recommendations in further research.

First, further research in evolving a way of writing better task descriptions in a Lean Design process is prohibited. Moreover, investigate the need and difference in the construction phase, in Lean Construction. It will be beneficial using experience from both User stories from Scrum and how to write sound requirements from RE.
	 
Second, the results indicate a considerable benefit in utilizing cloud technology in BIM. The personalized tools, however, addresses a different approach, namely cutting the exportation barrier, connecting the already familiar tools directly to a shared cloud database. Thus, utilizing both the interaction benefit of cloud BIM, with the powerful tools used by the modelers. This enhancement implicates further development of the technology as well as research into the actual effect.

\cleardoublepage